\chapter{Conclusioni e sviluppi futuri}

Pur essendo nato nel 2011, il network Litecoin ha recentemente guadagnato valore ed attenzione da parte di investitori e sviluppatori. La crescita è stata esponenziale soprattutto negli ultimi due anni e l’interesse allo sviluppo software per la blockchain stessa e la sua analisi si sta solo ora avvicinando a quello di Bitcoin. Le sue caratteristiche lo rendono adatto a pagamenti elettronici e correntemente, dopo un tentativo con LitePay, la fondazione Litecoin sta valutando l’adozione di una carta di debito simile ad un bancomat per pagamenti “fisici”, per cui c’è da attendersi un incremento di interesse dal lato sia economico che software.

Per quanto riguarda il tool, si può pensare ad un raffinamento delle analisi tramite adeguamento delle librerie già esistenti ai nuovi protocolli e in modo analogo si possono implementare parti software per effettuare analisi simili su hard forks di Bitcoin come Bitcoin Cash. Quest’ultima seppur recente (nata il 1 agosto 2017) ha come obiettivo la riduzione delle fees e l’aumento della velocità di validazione delle transazioni sul network, ragioni per cui c’è da attendersi esiti e sviluppi simili a quelli di Litecoin.


