\chapter{Conclusioni e sviluppi futuri}

\section{Conclusioni}
Pur essendo nato nel 2011, il network Litecoin ha recentemente guadagnato valore ed attenzione da parte di investitori e sviluppatori. La crescita è stata esponenziale soprattutto negli ultimi due anni e l’interesse allo sviluppo software per la blockchain stessa e la sua analisi si sta solo ora avvicinando a quello di Bitcoin. Le sue caratteristiche lo rendono adatto a pagamenti elettronici e correntemente, dopo un tentativo con LitePay, la fondazione Litecoin sta valutando l’adozione di una carta di debito simile ad un bancomat per pagamenti “fisici”, per cui c’è da attendersi un incremento di interesse dal lato sia economico che software.
Seppur ancora distante dalla diffusione di Bitcoin si possono riscontrare in Litecoin diversi


\section{Sviluppi futuri}

Per quanto riguarda il tool, si può pensare sia ad un raffinamento delle analisi tramite adeguamento delle librerie già esistenti ai nuovi protocolli che all'implementazione di nuovi componenti per effettuare analisi analoghe a quelle che ho appena svolto su hard forks di Bitcoin come Bitcoin Cash. Quest’ultima seppur recente (nata il 1 agosto 2017) ha come obiettivo la riduzione delle fees e l’aumento della velocità di validazione delle transazioni sul network, ragioni per cui c’è da attendersi esiti e sviluppi simili a quelli di Litecoin.
Le tracce per il raffinamento delle analisi in un futuro prossimo sono tante: ad esempio il miglioramento delle API dei servizi web di terze parti, spesso ancora un po' carenti per Litecoin, e la maggior diffusione dei dati storici ad esso relativi permetterebbero analisi più accurate e significative riguardanti i bilanci, i tassi di scambio e le fees in relazione all'euro.
